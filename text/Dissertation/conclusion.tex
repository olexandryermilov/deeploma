\chapter*{ВИСНОВКИ}                       % Заголовок
\addcontentsline{toc}{chapter}{ВИСНОВКИ}  % Добавляем его в оглавление
\section*{Результат}
Було розроблено застосунок, що є простим прототипом персонального асистенту. Застосунок є швидким, має різноманітну функціональність. Окрім виконання функції чат-боту, застосунок вміє збирати датасети для покращення своєї роботи. 

Розроблений застосунок має архітектуру, що спрощує його розширення та збільшення можливих функцій бота. 

Вдалося досягти виконання усіх поставлених задач
\begin{itemize}
\item Бот працює у месенджері Телеграм.
\item Бот має кілька закодованих реплік для простого спілкування з користувачем.
\item Бот може розпізнавати тип повідомлення з точність 91,1\%.
\item Бот вміє робити нагадування, витягуючи точну дату нагадування та тіло нагадування з тексту користувача. Бот збирає окремий датасет для витягування тіла нагадування з вихідного тексту.
\item Бот використовує інформацію з двох сторонніх сервісів. Додавання інших сторонніх сервісів є простою задачею і займає кілька рядків коду.
\item Бот розпізнає теми запитів до нього.
\item Бот може розпізнавати іменовані сутності і запитує більше інформації про них.
\item Бот аналізує загальний тон повідомлення, та має спеціальні репліки, що залежать від настрою повідомлень користувача.
\item Бот може відповідати на запитання, використовуючи одну із найкращих моделей для SQuAD датасету.
\item Бот зберігає усю інформацію надану користувачем
\item Бот збирає два окремих датасети, що можуть допомогти у вдосконаленні його роботи в майбутньому.
\item Більшість відповідей на повідомлення вкладаються у час до 3 секунд. Єдиним повільним місцем, що потенційно може зайняти більше ніж 5 секунд - відповідь на питання по великому тексту.
\end{itemize}

Було зроблено короткий огляд найважливіших задач сфери обробки природньої мови. Описано спосіб оцінювання моделей, що вирішують ці задачі. Використано моделі, що дають найкращі результати у наведених вище задач. На прикладі боту можна побачити, як ці задачі використовуються у реальних застосунках . 

Також зроблено теоретичний огляд моделей та засобів розробки, що використовувались при створенні додатку. Розглянуто роботу моделі BERT, та алгоритм дерев рішень.

Зроблено аналіз застосунків, схожих за метою, наведено їх коротке порівняння та опис переваг.

Користувач може користуватися функцією відповіді на питання у багатьох різних цілях - опрацювання тексту, підготовки до тестів та контрольних, тощо. Цією ж функцією може користуватися бот для більшої точності  та парсингу запитів до нього, тощо.

Дана робота може слугувати базою для майбутніх покращень, що зроблять досвід роботи користувача з ботом більш гладким та персоналізованим.


\section*{Можливі покращення} 
\begin{itemize}
\item Застосування патерну event-sourcing, що полягає у зберіганні всіх подій користувача та побудові інформації про нього як агрегацію усіх збережених подій. Таким чином, ми зможемо змінювати реакції на події все після того, як відбудеться сама подія, що ще більше спростить розширюваність додатку.
\item Для розпізнавання типу повідомлення можна додати парсинг частин мови та частин речення накшталт nltk. Більшість повідомлень складаються не більш ніж з одного речення, а вищевказана класифікація типів повідомлень майже відтворює класифікацію типів речень у англійській мові. А для речень за допомогою парсингу частин речення, мови та побудови дерева залежності слів можна майже з стовідсотковою точністю визначити, чи воно є питанням, чи ні. Скомбінувавши цей спосіб з класифікатором, можна підвищити точність визначення типу повідомлення. 
\item Для пришвидшення швидкості відповіді на питання варто скористатись GPU замість CPU.
\item Для збільшення привабливості користування ботом, необхідно розширювати його функціонал. У цьому можуть допомогти безкоштовні API та вільно поширюванні бібліотеки. 
\item Для збільшення точності класифікації типу повідомлення варто спробувати метод ngram, оскільки він враховує не лише які саме слова наявні у повідомленні, а і їх порядок, що дає змогу краще відрізняти питальні від стверджувальних повідомлень.
\item Для моделей, запущених на Python, можна вдосконалити код, що відповідає на взаємодію з серверною частиною додатку. Якщо використати більш швидкі фреймворки для виставляння REST-endpoint'ів, можна пришвидшити деякі частини застосунку.
\item BERT погано вміє відповідати на питання типу "Так/Ні", але ми можемо передобробити такі питання і заміняти їх на wh-питання, після чого декодувати відповіді у Так чи Ні.
\item Для пришвидшення роботи бота у майбутньому, можна обробляти запити від різних користувачів параллельно. Оскільки в данній моделі бота дані про ріних користувачів не перетинаються, параллельно обробка різних запитів жодним чином не зашкодить цілісності даних.
\item Використання бота у окремому застосунку дозволить отримувати більше данних про користувача.  Наприклад, можна використовувати геолокацію, модель телефона, тощо. Це дозволить значно розширити функції - як приклад, можна створити нагадування, що залежать від локації: "Нагадай мені купити молоко коли я буду в магазині."
\item Використовуючи алгоритми генерації мови, можемо генерувати тексти у випадках, коли не потрібно виконувати запит чи відповідати на питання. Це дасть змогу використовувати бота не тільки як помічника, але і  як співрозмовника. 
\item Можна розпізнавати голосові повідомлення користувача, та в цілому більш широко використовувати можливості платформи Telegram - надсилання відео, зображень, тощо.
\end{itemize}
