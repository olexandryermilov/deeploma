\chapter*{ВСТУП}
\addcontentsline{toc}{chapter}{ВСТУП}
\textbf{Оцінка сучасного стану розробки.} 

Технології обробки та генерації природньої мови останнім часом стрімко розвиваються. Це пов'язано як з збільшенням потужності комп'ютерів, так і з здешевшанням зберігання величезних масивів даних, що дає можливість тренувати кращі моделі на більших датасетах. Нові можливості у даній сфері, в свою чергу, надихнули розробників створювати чат-боти, як засіб для підтримки бізнесу. Окремою гілкою розвитку чат-ботів є персональні асистенти, які більше фокусуються на підтримці діалогу та виконанні простих задач для користувача, аніж на наданні інформації та проведенні користувача через заплановані сценарії.


\textbf{Актуальність роботи та підстави для її виконання.}

Робота над обробкою природньої мови є надзвичайно актуальною. Лише нещодавно, у 2019 р., компанія Google представила нову модель BERT, що показала найкращі результати у вирішенні кількох типів задач, серед яких задачі GLUE , SQuAD і SWAG. 


\textbf{Мета й завдання роботи.} 

Завданням роботи є створення застосунку, який можна використовувати у якості персонального асистента - тобто він має задовільняти наступним критеріям: 
\begin{itemize}
\item Зберігає контекст користувача з яким спілкується - знає основні факти про нього та може відповісти на прості питання про користувача (за умови що користувач надає таку інформацію). 
\item Може виконувати деякі прості задачі - ставити нагадування, тощо.
\item Є простим для розширення та підтримки існуючого коду, причому розширення можливі як в сторону збільшення кількості функцій боту, так і даних що він зберігає про даного користувача.
\end{itemize}

 Метою роботи є огляд та застосування сучасних засобів розробки та сучасних моделей для роботи з природними мовами. 
 
 
\textbf{ Об'єкт, методи й засоби розроблення}

Об'єктом даної роботи є створення додатку, що може бути використаним як персональний асистент.


Розробленню передував детальний аналіз існуючих рішень, кращих практик розробки, проектування архітектури з врахуванням многошаровості додатку, підбір оптимальних моделей та бібліотек, що допомагають у вирішення завдання роботи. Розробка була ітеративною, кожна наступна функція тестувалась як окремо, так і у поєднанні з існуючими.


Під час розробки було використано мови програмування Scala та Python, а також ряд вільнопоширюванних бібліотек та пакетів для вищезазначених мов: Apache Spark - бібліотека для обробки великих об’ємів даних, Spark NLP - бібліотека для попередньої обробки текстів, Specs2 - бібліотека для тестування коду на Scala, numpy, pandas - бібліотеки для роботи з даними, nltk - бібліотека для роботи з природніми мовами, sklearn - бібліотека для тренування моделі, transformers - бібліотека з готовими моделями для роботи з текстами. . В якості інструменту створення програмного засобу було обрано Intellij IDEA – інтегроване середовище розробки яке підтримує різноманітні мови розробки, яке є безкоштовним, вільно поширюваним, з відкритим вихідним кодом, для роботи з кодом на Python було обрано Jupyter Notebooks як зручний інструмент для швидкої ітераитвної розробки та тренування моделей. Scala, Python та обрані бібліотеки як надають багатий набір методів для машинного навчання, так і є зручними і надійними мовами для створення додатків.. 


\textbf{Можливі сфери застосування.}

Даний додаток може застосовуватись як персональний асистент з невеликою кількістю допоміжних функцій. Крім того, враховуючи легкість розширення, нескладно збільшувати перелік функцій, а отже і список можливих сфер застосування. Так, чат-боти з технологіями генерації природньої мови можуть стати в нагоді для людей, що через карантин або з інших причин почувають себе самотньо, тощо.   


\textbf{Взаємозв’язок з іншими роботами.}

Дана робота базується на останніх дослідженнях у сфері обробки природньої мови. 
